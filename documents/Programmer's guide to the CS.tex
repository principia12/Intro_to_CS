\documentclass[twoside]{article}

\usepackage{lipsum} % Package to generate dummy text throughout this template

\usepackage[sc]{mathpazo} % Use the Palatino font
\usepackage[T1]{fontenc} % Use 8-bit encoding that has 256 glyphs
\linespread{1.05} % Line spacing - Palatino needs more space between lines
\usepackage{microtype} % Slightly tweak font spacing for aesthetics

\usepackage[hmarginratio=1:1,top=32mm,columnsep=20pt]{geometry} % Document margins
\usepackage{multicol} % Used for the two-column layout of the document
\usepackage[hang, small,labelfont=bf,up,textfont=it,up]{caption} % Custom captions under/above floats in tables or figures
\usepackage{booktabs} % Horizontal rules in tables
\usepackage{float} % Required for tables and figures in the multi-column environment - they need to be placed in specific locations with the [H] (e.g. \begin{table}[H])
\usepackage{hyperref} % For hyperlinks in the PDF

\usepackage{lettrine} % The lettrine is the first enlarged letter at the beginning of the text
\usepackage{paralist} % Used for the compactitem environment which makes bullet points with less space between them

\usepackage{kotex}
\usepackage{braket}
\usepackage{array}
\usepackage{calc}
\usepackage{graphicx}
\usepackage{listings}


\lstset{frame=tb,
  language=lisp,
  aboveskip=3mm,
  belowskip=3mm,
  showstringspaces=false,
  columns=flexible,
  basicstyle={\small\ttfamily},
  numbers=none,
  numberstyle=\tiny\color{gray},
  keywordstyle=\color{blue},
  commentstyle=\color{dkgreen},
  stringstyle=\color{mauve},
  breaklines=true,
  breakatwhitespace=true
  tabsize=3}


\lstdefinelanguage{JavaScript}{
  keywords={typeof, new, true, false, catch, function, return, null, catch, switch, var, if, in, while, do, else, case, break},
  keywordstyle=\color{blue}\bfseries,
  ndkeywords={class, export, boolean, throw, implements, import, this},
  ndkeywordstyle=\color{darkgray}\bfseries,
  identifierstyle=\color{black},
  sensitive=false,
  comment=[l]{//},
  morecomment=[s]{/*}{*/},
  commentstyle=\color{purple}\ttfamily,
  stringstyle=\color{red}\ttfamily,
  morestring=[b]',
  morestring=[b]"
}

  
  
\lstnewenvironment{C}
  {\lstset{
	language=C
}}
  {}

\lstnewenvironment{Java}
  {\lstset{
	language=Java
}}
  {}

\lstnewenvironment{Python}
  {\lstset{
	language=Python
}}
  {}

\lstnewenvironment{JavaScript}
  {\lstset{
	language=JavaScript
}}
  {}

\lstnewenvironment{HTML}
  {\lstset{
	language=HTML
}}
  {}

  
  
\usepackage{color}
\usepackage[table,xcdraw]{xcolor}
\usepackage{adjustbox}


\definecolor{dkgreen}{rgb}{0,0.6,0}
\definecolor{gray}{rgb}{0.5,0.5,0.5}
\definecolor{mauve}{rgb}{0.58,0,0.82}



\hypersetup{%
    pdfborder = {0 0 0}
}



\usepackage{abstract} % Allows abstract customization
\renewcommand{\abstractnamefont}{\normalfont\bfseries} % Set the "Abstract" text to bold
\renewcommand{\abstracttextfont}{\normalfont\small\itshape} % Set the abstract itself to small italic text

\usepackage{titlesec} % Allows customization of titles
%\renewcommand\thesection{\Roman{section}} % Roman numerals for the sections
\renewcommand\thesubsection{\Roman{subsection}} % Roman numerals for subsections
\titleformat{\section}[block]{\large\scshape\centering}{\thesection.}{1em}{} % Change the look of the section titles
\titleformat{\subsection}[block]{\large}{\thesubsection.}{1em}{} % Change the look of the section titles

\usepackage{fancyhdr} % Headers and footers
\pagestyle{fancy} % All pages have headers and footers
\fancyhead{} % Blank out the default header
\fancyfoot{} % Blank out the default footer
\fancyhead[C]{ Programmer's guide to the CS} % Custom header text
\fancyfoot[RO,LE]{\thepage} % Custom footer text

\setcounter{section}{-1}
\setlength\parindent{0pt}

%----------------------------------------------------------------------------------------
%	TITLE SECTION
%----------------------------------------------------------------------------------------

\begin{document}
\begin{titlepage}

\newcommand{\HRule}{\rule{\linewidth}{0.5mm}} % Defines a new command for the horizontal lines, change thickness here

\center % Center everything on the page
 
%----------------------------------------------------------------------------------------
%	HEADING SECTIONS
%----------------------------------------------------------------------------------------

\vspace*{3cm}
\textsc{\Large CS042}\\[0.5cm] % Major heading such as course name
\textsc{\large Scalable Data Structure Study}\\[0.5cm] % Minor heading such as course title

%----------------------------------------------------------------------------------------
%	TITLE SECTION
%----------------------------------------------------------------------------------------

\HRule \\[0.4cm]
{ \huge \bfseries Programmer's guide to the CS}\\[0.4cm] % Title of your document
\HRule \\[1.5cm]
 
%----------------------------------------------------------------------------------------
%	AUTHOR SECTION
%----------------------------------------------------------------------------------------

\begin{minipage}{0.4\textwidth}
\begin{flushleft} \large
\emph{Author:}\\
Seungwoo \textsc{Schin} \\ % Your name
Gihyo \textsc{Jang} \\ % Your name
Seowon \textsc{Oh} % Your name
\end{flushleft}
\end{minipage}
\begin{minipage}{0.4\textwidth}
\begin{flushright} \large
\emph{Typeset by:} \\
Seungwoo \textsc{Schin} % Supervisor's Name
\end{flushright}
\end{minipage}\\[4cm]

% If you don't want a supervisor, uncomment the two lines below and remove the section above
%\Large \emph{Author:}\\
%John \textsc{Smith}\\[3cm] % Your name

\textsc{ PyDal study fork(); }\\[1.5cm] % Name of your university/college

%----------------------------------------------------------------------------------------
%	DATE SECTION
%----------------------------------------------------------------------------------------

{\large \today}\\[3cm] % Date, change the \today to a set date if you want to be precise
%2015 Spring Semester

%----------------------------------------------------------------------------------------
%	LOGO SECTION
%----------------------------------------------------------------------------------------

%\includegraphics{Logo}\\[1cm] % Include a department/university logo - this will require the graphicx package
 
%----------------------------------------------------------------------------------------

%\vfill % Fill the rest of the page with whitespace

\end{titlepage}

% Table of contents 

\tableofcontents
\newpage


\section{Introduction}

\subsection{Administrative Details}

기본적인 데이터구조들을 여러 언어로 짜보고, 코딩한 결과를 서로 비교해 보는 걸 목적으로 합니다. 다른 스터디에서도 사용할 수 있도록 작성하였습니다. 기본적인 규칙들은 다음과 같습니다. 

\begin{enumerate}

\item 전반적인 공지나 질문 등은 카톡방을 통해서 공유합니다. 
\item 사용하는 언어는 자유이나, file io가 가능해야 합니다. 재귀, loop 등을 사용할 수 있으면 좋습니다. 

\item 1주일에 한번씩 매주 토요일 10시부터 12시 반까지 만납니다. 

\item 만날 때마다 전 주에 정해진 데이터구조에 대해서 한 명이 공부해서 아래 일들을을 하셔야 합니다.

\begin{enumerate}
\item 데이터구조에 대한 간략한 설명 
\item 데이터구조를 txt 파일로 표현하는 방법에 대한 표준 설정 
\item 해당 데이터구조에 대한 operation의 결과를 txt 파일로 나타내는 방법에 대한 표준 설정 
\item 테스트용 파일과 솔루션 파일 작성
\end{enumerate}

\item 만난 후 다음번에 만나기 전까지, 해당 스터디에서 커버한 부분에 대한 코드를 작성한 후, 테스트 결과와 같이 올려주세요. 

\item 스터디장은 위 코드들과 테스트, 솔루션 파일들을 받아서 github에 업로드해야 하며, 스터디를 진행할 때마다 커버한 부분을 \LaTeX 으로 작성해서 올려야 합니다. 

\item 스터디장은 \LaTeX을 할 줄 알아야 합니다. 

\end{enumerate}

\subsubsection{사용할 언어들}

사용 언어는 자유이며, 아래의 언어는 사용 가능한 언어들의 hello world! 예제입니다. 아래에 없는 언어를 사용하고 싶으시면 스터디장에게 말해서 언어를 추가해 주세요. 

\begin{enumerate}

\item C


\begin{C}
#include <stdio.h> 

int main()
{
	printf("Hello World!\n"); 
}
\end{C}

\item Java


\begin{Java}
public class Foo {
  public static void main(String args[]) {
    System.out.println("for test");
  }
}
\end{Java}

\item{Python}

\begin{Python}
print "hello world!"
\end{Python}

\item JavaScript

\begin{JavaScript}
// Hello in Javascript

// display prompt box that ask for name and 
// store result in a variable called who
var who = window.prompt("What is your name");

// display prompt box that ask for favorite color and 
// store result in a variable called favcolor
var favcolor = window.prompt("What is your favorite color");

// write "Hello" followed by person' name to browser window
document.write("Hello " + who);

// Change background color to their favorite color
document.bgColor = favcolor;
\end{JavaScript}

\end{enumerate}

\subsection{Programming Basics}

프로그래밍 언어와 독립적으로, 알고리즘을 평가하거나 디자인하는 방법에 대한 항목들입니다. 

\subsubsection{Timal/Spatial complexity} 

O(n)

\subsubsection{Recursion}

definition
complexity 
Recursion example 
vs iteration 
stack overflow / tail-call recursion (+ continuation?)


\subsubsection{Divide and Conquer}

definition
example 


\subsubsection{Dynamic Programming}

introcd ../
example
vs greedy 


\newpage



\section{List}


\subsection{Structures to implement}

\subsubsection{Singly linked list}

\subsubsection{Doubly linked list}

\subsubsection{Circular linked list}

\subsection{Operations to implement}

\subsubsection{Sorting}

\subsubsection{Searching}

\subsubsection{Insertion}

\subsubsection{Deletion}

\newpage
\section{Stack}


스택은 데이터구조 중 나중에 들어온 것이 먼저 나오는(LIFO - Last In, First Out) 자료구조를 스택이라고 합니다. 스택은 여러 방법으로 구현될 수 있으며, 구현에 따라서 performance 또한 달라집니다. 예를 들어서, array를 이용하여 구현한 스택과 linked list를 이용하여 구현한 스택, 그리고 BST를 이용하여 구현한 스택은 모두 performance가 다르게 됩니다. array의 경우 array의 끝에 자료를 추가하거나 지우는 것은 O(1)의 시간이 걸리지만, linked list의 경우 구현에 따라서 O(1)이 될수도, O(n)이 될 수도 있습니다. BST의 경우 언제나 O(log n)\footnote{Balanced된 경우의 worst case} 이 됩니다. 

스택에서 가장 중요한 operation은 push와 pop입니다. push는 스택 구조에 데이터를 넣는 것을 말하고, pop은 스택에서 마지막에 들어간 자료를 제거하는 것을 말합니다. 

\subsection{Structures to implement}

\subsubsection{Simple stack}

\begin{compactitem}

\item Array Implementation 

\begin{Python}
class Stack(): 
    def __init__(self, contents):
        self.contents = contents
        
    def pop(self):
        if self.contents:
            return self.contents[-1]
        else:   
            return None
    def push(self, elem):
        self.contents.append(elem)
\end{Python}
위 코드는 파이썬이고, 파이썬의 list는 array of pointer로 구현되어 있으므로\footnote{http://www.laurentluce.com/posts/python-list-implementation/}  위 구현은 array를 이용한 스택의 구현입니다. 

\item Linked List Implementation

\begin{Python}
class LinkedNode():
    def __init__(self, next_node = None, data = None):
        self.next_node = next_node
        self.data = data
        
    def __str__(self):
        if self.next_node == None:
            return self.data
        else:
            return self.data + self.next_node.__str__()
    
    def pop(self):
        if self.next_node.next_node == None:
            res = self.next_node
            self.next_node = None
            return res
        else: 
            return self.next_node.pop()
    def push(self, data):
        if self.next_node.next_node == None:
            self.next_node.next_node = LinkedNode(None, data)
\end{Python}


\end{compactitem}

\subsection{Operations to implement}

위 코드에 이미 구현되어 있으므로 아래 코드는 생략한다. 
\subsubsection{Insertion}
Push operation 

\subsubsection{Deletion}
pop operation 

\newpage

\section{Queue}


\subsection{Structures to implement}

I learned how to implement queue! 

\subsubsection{Simple queue}

\subsubsection{Priority queue}

\subsubsection{Dequeue (double ended queue)}

\subsection{Operations to implement}

\subsubsection{Sorting}

\subsubsection{Searching}

\subsubsection{Insertion}

\subsubsection{Deletion}

\newpage

\section{Advanced Structure}


\subsection{Structures to implement}

\subsubsection{Tree}

\paragraph{Simple binary tree}
\paragraph{Full binary tree }
\paragraph{Perfect binary tree}
\paragraph{Complete binary tree}
\paragraph{Binary search tree}
\paragraph{AVL tree}
\paragraph{Red-black tree}
\paragraph{Heap tree}
\paragraph{B-tree}
\paragraph{ 2-3-4 tree}
\paragraph{B+ tree}


\subsubsection{Graph}
\paragraph{Simple graph}

\paragraph{Directed graph}

\paragraph{Directed weighted graph}


\subsection{Operations to implement}

\subsubsection{Elementary algorithms}

\paragraph{BFS}

\paragraph{DFS}

\paragraph{Topological sort}

\subsubsection{Minimum spanning tree}

\paragraph{Prim}

\paragraph{Kruskal}


\subsubsection{Shortest path algorithms}

\paragraph{Bellman-Ford algorithms}
\paragraph{Dijkstra's algorithm}
\paragraph{Floyd-Warshall algorithm}
\paragraph{Johnson's algorithm}


\newpage

\end{document}



























